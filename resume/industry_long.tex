% https://www.overleaf.com/latex/templates/cis-grad-template-dev/pjcttwysqvym


% Use the custom resume.cls style
\documentclass{resume}

% Document margins
\usepackage[top=0.15in, bottom=0.15in, right=0.4in, left=0.3in]{geometry}

\name{Wei-Chih Huang}
\address{
\href{mailto:noctildon2@gmail.com}{Email} \\
\href{https://www.linkedin.com/in/wei-chih-huang-38b075233/}{Linkedin} \\
\href{https://github.com/noctildon}{Github} \\
\href{https://noctildon.github.io}{Personal Website}}

\begin{document}


\vspace{-2.75em}
\begin{rSection}{Education}
{\bf PhD in Physics}, Texas A\&M University, US \hfill {Aug 2019 - Aug 2025 (expected)} \\
{\bf BS in Physics}, National Tsing Hua University, Taiwan \hfill {Aug 2015 - Jun 2019}
\vspace{-0.5em}


\begin{rSection}{Projects}
    \vspace{-1.25em}
    \item \textbf{PyBigstick} - {NumPy, Pandas, Matplotlib, Streamlit, Docker} \hfill {\href{https://github.com/noctildon/pyBigstick}{github}}
        \begin{itemize}
        \itemsep -3pt {}
        \item Saved 95\% of time writing input scripts for \href{https://github.com/cwjsdsu/BigstickPublick}{BIGSTICK} (Large Scale Nuclear Shell Model Code)
        \item Analyzed any nucleus and predict experimental outcomes with at least 60\% accuracy
        \item Used Streamlit and Docker to create an interactive data dashboard on any platform
        \end{itemize}
    \item \textbf{Pro Cyclists Race Analysis} - {NumPy, Pandas, BeautifulSoup, scikit-learn, XGBoost, Pytorch, Runpod} \hfill {\href{https://github.com/noctildon/pro_cyclists}{github}}
            \begin{itemize}
            \itemsep -3pt {}
            \item Implemented high performance multi-threading web scraping script by BeautifulSoup (5 times faster)
            \item Preprocessed the data (clean, format, normalize) with NumPy, Pandas, SciPy, and scikit-learn
            \item Made the prediction with 20\% better performance than a trivial model with scikit-learn, XGBoost, and Pytorch
            \item Deployed the data and model to \href{https://www.runpod.io/}{Runpod} (GPU cloud service) for training and saved 80\% costs
            \end{itemize}
    \item \textbf{Aggie Job Referral} - {Django, SQLite, PostgreSQL, Heroku, Bootstrap} \hfill {\href{https://github.com/noctildon/aggie-job-referral}{github}}
        \begin{itemize}
        \itemsep -3pt {}
        \item Built a referral website to reduce the time of networking by 40\%
        \item Deployed to Heroku with specially designed PostgreSQL database schema to save the disk space by 20\%
        \end{itemize}
    \item \textbf{Curve Fitting GUI} - {SciPy, NumPy, Matplotlib, PyQT} \hfill {\href{https://github.com/noctildon/curve_fitting}{github}}
        \begin{itemize}
        \itemsep -3pt {}
        \item User friendly graphical user interface tool for curve fitting
        \end{itemize}
\end{rSection}



\begin{rSection}{Research Experience}
\vspace{-1.25em}
\item \textbf{Inelastic Neutrino/Dark Matter - Nucleus Scattering by BIGSTICK} \hfill {\href{https://github.com/noctildon/Inelastic}{github}}
    \begin{itemize}
    \itemsep -3pt {}
    \item Parallelized and compiled BIGSTICK with MPI/OpenMP in computer cluster
    \item Did the statistical analysis on the large multi-dimensional outputs by Python and Mathematica
    \item Published 3 papers and presented several successful talks at workshops
    \end{itemize}
\item \textbf{Searching for Axion/Dark matter in High Energy Physics Experiments} \hfill {\href{https://github.com/noctildon/alplib}{axion}}, {\href{https://github.com/noctildon/lightDM}{dark matter}}
    \begin{itemize}
    \itemsep -3pt {}
    \item Construct analytical models for axion and dark matter, and automated the statistical analysis with Python
    \item Used Python multiprocessing and function caching to speed up the numerical analysis by 1000 times on average
    \end{itemize}
\item \textbf{Inflation and Late-time Acceleration in a New Gravity Theory}
    \begin{itemize}
    \itemsep -3pt {}
    \item Created time-dependent partial differential equations to describe the features of the universe
    \item Programed Mathematica and Python to stimulate and visualize the evolution of the universe
    \end{itemize}
\item \textbf{Dark Matter in Merging Galaxies}
    \begin{itemize}
    \itemsep -3pt {}
    \item Automated the analysis process of dark matter near a galaxy with CASA (data processing software for radio telescopes arraies, written in IPython)
    \end{itemize}
\item \textbf{Application of Deep Learning in AdS/CFT}
    \begin{itemize}
    \itemsep -3pt {}
    \item Integrated deep learning with Ads/CFT (a well-known theory in high energy physics) \href{https://noctildon.github.io/files/pdf/archive/DL.pdf}{text}
    \end{itemize}
\item \textbf{Coherent Elastic neutrino-nucleus Scattering (CE$\nu$NS): Sterile Neutrino Search}
    \begin{itemize}
    \itemsep -3pt {}
    \item Construct a statistics model for sterile neutrino
    \end{itemize}
\end{rSection}


\begin{rSection}{Publication}
    \begin{itemize}
        \item Probing the dark sector with nuclear transition photons \hfill \href{https://arxiv.org/pdf/2302.10250.pdf}{arxiv}\\
        {\footnotesize \it Bhaskar Dutta, \textbf{Wei-Chih Huang}, Jayden L. Newstead}
        \item Inelastic nuclear scattering from neutrinos and dark matter \hfill \href{https://arxiv.org/pdf/2206.08590.pdf}{arxiv}\\
        {\footnotesize \it Bhaskar Dutta, \textbf{Wei-Chih Huang}, Jayden L. Newstead, Vishvas Pandey}
        \item Axion-Like Particle Production at Beam Dump Experiments with Distinct Nuclear Excitation Lines \hfill \href{https://arxiv.org/pdf/2207.13659.pdf}{arxiv}\\
        {\footnotesize \it Loyd Waites, Adrian Thompson, Adriana Bungau, Janet M. Conrad, Bhaskar Dutta, \textbf{Wei-Chih Huang}, Doojin Kim, Michael Shaevitz, Joshua Spitz}
    \end{itemize}
\end{rSection}


\begin{rSection}{Extra-Curricular Activities}
    \begin{itemize}
        \item Project Manager at \href{https://aggiecodingclub.com/}{Aggie Coding Club} \hfill 2022
        \item Data Science Ambassador representing Physics Department at Texas A\&M \href{https://noctildon.github.io/DS_ambassador/index.html}{webpage} \hfill 2022 - 2023
    \end{itemize}
\end{rSection}



%------------------------
% Use this more detailed section if you have Relevant work experience
% keep your resume to 1 page, if you need to remove a project to display relevant experience
% that is okay
% ----------------------------
% \begin{rSection}{EXPERIENCE}

% \textbf{Role Name} \hfill Jan 2017 - Jan 2019\\
% Company Name \hfill \textit{San Francisco, CA}
%  \begin{itemize}
%     \itemsep -3pt {}
%      \item Achieved X\% growth for XYZ using A, B, and C skills.
%      \item Led XYZ which led to X\% of improvement in ABC
%     \item Developed XYZ that did A, B, and C using X, Y, and Z.
%  \end{itemize}

% \textbf{Role Name} \hfill Jan 2017 - Jan 2019\\
% Company Name \hfill \textit{San Francisco, CA}
%  \begin{itemize}
%     \itemsep -3pt {}
%      \item Achieved X\% growth for XYZ using A, B, and C skills.
%      \item Led XYZ which led to X\% of improvement in ABC
%     \item Developed XYZ that did A, B, and C using X, Y, and Z.
%  \end{itemize}

% \end{rSection}


\begin{rSection}{Honors and Awards}
    \begin{itemize}
        \item \textbf{Data Science Ambassador Scholarship}\hfill 2022 - 2023\\
        Data Science Ambassador Scholarship Program at Texas A\&M Institute of Data Science
        \item \textbf{Three Years Tsing Hua University Scholarship (2\% acceptance rate)} \hfill 2015 - 2018 \\
        Tuition wavier plus accommodation and textbooks subsidy
        \item \textbf{Undergraduate Research Scholarship} \hfill Fall 2018 \\
        The scholarship for the New Gravity Theory
    \end{itemize}
\end{rSection}


\begin{rSection}{Teaching Experience}
    \begin{itemize}
        \item \textbf{Lecturer} {{\it Texas A\&M Physics Department} Data Science in Physics} \hfill 2022 - 2023
        \item \textbf{Teaching Assistant} {{\it Texas A\&M} Thermodynamics and Statistical Mechanics} \hfill Fall 2019
        \item \textbf{Teaching Assistant} {{\it Texas A\&M} Electricity and Magnetism for Engineering and Science (Lab)} \hfill Summer 2020
        \item \textbf{Teaching Assistant} {{\it Texas A\&M} Newtonian Mechanics for Engineering and Science} \hfill 2020 - 2022
        \item \textbf{Teaching Assistant} {{\it Texas A\&M} Electricity and Magnetism for Engineering and Science} \hfill 2020 - 2022
    \end{itemize}
\end{rSection}


\end{rSection}


%------------------------
% Use this more detailed section if you have Relevant work experience
% keep your resume to 1 page, if you need to remove a project to display relevant experience
% that is okay
% ----------------------------
% \begin{rSection}{EXPERIENCE}

% \textbf{Role Name} \hfill Jan 2017 - Jan 2019\\
% Company Name \hfill \textit{San Francisco, CA}
%  \begin{itemize}
%     \itemsep -3pt {}
%      \item Achieved X\% growth for XYZ using A, B, and C skills.
%      \item Led XYZ which led to X\% of improvement in ABC
%     \item Developed XYZ that did A, B, and C using X, Y, and Z.
%  \end{itemize}

% \textbf{Role Name} \hfill Jan 2017 - Jan 2019\\
% Company Name \hfill \textit{San Francisco, CA}
%  \begin{itemize}
%     \itemsep -3pt {}
%      \item Achieved X\% growth for XYZ using A, B, and C skills.
%      \item Led XYZ which led to X\% of improvement in ABC
%     \item Developed XYZ that did A, B, and C using X, Y, and Z.
%  \end{itemize}

% \end{rSection}


% \begin{rSection}{SKILLS}
% \begin{tabular}{ @{} >{\bfseries}l @{\hspace{6ex}} l }
% Languages & Python, Bash, Mathematica \\
% Tools & Git, Docker, AWS, Django, Pandas, Numpy, SciPy\\
% \\
% \end{tabular}\\
% \end{rSection}


\end{document}
