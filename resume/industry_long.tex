% https://www.overleaf.com/latex/templates/cis-grad-template-dev/pjcttwysqvym


% Use the custom resume.cls style
\documentclass{resume}

% Document margins
\usepackage[top=0.15in, bottom=0.15in, right=0.4in, left=0.3in]{geometry}

\name{Wei-Chih Huang}
\address{
\href{mailto:s104021230@tamu.edu}{Email} \\
\href{https://www.linkedin.com/in/wei-chih-huang-38b075233/}{Linkedin} \\
\href{https://github.com/noctildon}{Github} \\
\href{https://noctildon.github.io/about-me/about-me.html}{Personal Website}}

\begin{document}


\vspace{-2.75em}
\begin{rSection}{Education}
{\bf PhD in Physics}, Texas A\&M University, US \hfill {August 2019 - June 2025 (expected)} \\
{\bf BS in Physics}, National Tsing Hua University, Taiwan \hfill {August 2015 - June 2019}
\vspace{-0.5em}


\begin{rSection}{Projects}
    \vspace{-1.25em}
    % TODO: Add AWS after done
    \item \textbf{Aggie Job Referral} - {Django, SQLite, PostgreSQL, Heroku, Bootstrap} \hfill {\href{https://boiling-hollows-75833.herokuapp.com/}{website}}
        \begin{itemize}
        \itemsep -3pt {}
        \item Built a referral website to reduce the time of networking by 40\%
        \item Deployed to Heroku with specially designed PostgreSQL database schema to save the disk space by 20\%
        \end{itemize}
    \item \textbf{PyBigstick} - {NumPy, Pandas, Matplotlib, Streamlit, Docker} \hfill {\href{https://github.com/noctildon/pyBigstick}{github}}
        \begin{itemize}
        \itemsep -3pt {}
        \item Saved 95\% of time writing input scripts for \href{https://github.com/cwjsdsu/BigstickPublick}{BIGSTICK} (Large Scale Nuclear Shell Model Code)
        \item Able to analyze any nucleus and predict experimental outcomes with at least 60\% accuracy
        \item Created an interactive data dashboard with Streamlit
        \item Virtualized the app with Docker to run on any machine
        \end{itemize}
    \item \textbf{Pro Cyclists Race Analysis} - {NumPy, Pandas, SciPy, BeautifulSoup, scikit-learn, XGBoost, Pytorch} \hfill {\href{https://github.com/noctildon/pro_cyclists}{github}}
        \begin{itemize}
        \itemsep -3pt {}
        \item Implemented high performance multi-threading web scraping script by BeautifulSoup (5 times faster)
        \item Preprocessed the data (clean, format, normalize) with NumPy, Pandas, SciPy, and scikit-learn
        \item Used scikit-learn, XGBoost, and Pytorch to build linear, RF, DNN, RNN models and predict the race outcome, which is 20\% better than a trivial model
        \end{itemize}
    \item \textbf{Curve Fitting GUI} - {SciPy, NumPy, Matplotlib, PyQT} \hfill {\href{https://github.com/noctildon/curve_fitting}{github}}
        \begin{itemize}
        \itemsep -3pt {}
        \item User friendly graphical user interface tool for curve fitting
        \end{itemize}
\end{rSection}



\begin{rSection}{Research Experience}
\vspace{-1.25em}
\item \textbf{Inelastic Neutrino/Dark Matter - Nucleus Scattering by BIGSTICK}\hfill \href{https://arxiv.org/pdf/2206.08590.pdf}{arxiv}
    \begin{itemize}
    \itemsep -3pt {}
    \item Parallelized and compiled BIGSTICK with MPI/OpenMP in computer cluster
    \item Virtualized BIGSTICK with Docker to resolve the incompatibility with the cluster
    \item Did the statistical analysis on the multi-dimensional outputs by Python and Mathematica
    \item Published a paper and present several successful talks at workshops \href{https://noctildon.github.io/physics/Phenon_2022.pdf}{slides}, \href{https://noctildon.github.io/physics/Plains_2022.pdf}{slides}
    \end{itemize}
    \item \textbf{Searching for Axions in High Energy Physics Experiments}\hfill \href{https://arxiv.org/pdf/2207.13659.pdf}{arxiv}
    \begin{itemize}
    \itemsep -3pt {}
    \item Construct analytical models for axion (a theoretical particle) in the experiments
    \item Modularized and automated the statistical analysis
    \end{itemize}
\item \textbf{Inflation and Late-time Acceleration in a New Gravity Theory}
    \begin{itemize}
    \itemsep -3pt {}
    \item Created time-dependent partial differential equations to describe the features of the universe
    \item Programed Mathematica and Python to stimulate and visualize the evolution of the universe
    \end{itemize}
\item \textbf{Dark Matter in Merging Galaxies}
    \begin{itemize}
    \itemsep -3pt {}
    \item Automated the analysis process of dark matter near a galaxy with CASA (data processing software for radio telescopes arraies, written in IPython)
    \item Presented a talk at workshop \href{https://noctildon.github.io/physics/dm.pdf}{slides}
    \end{itemize}
\item \textbf{Application of Deep Learning in AdS/CFT} \hfill \href{https://noctildon.github.io/physics/DL.pdf}{text}
    \begin{itemize}
    \itemsep -3pt {}
    \item Integrated deep learning with Ads/CFT (a well-known theory in high energy physics)
    \end{itemize}
\item \textbf{Coherent Elastic neutrino-nucleus Scattering (CE$\nu$NS): Sterile Neutrino Search}
    \begin{itemize}
    \itemsep -3pt {}
    \item Construct a statistics model for sterile neutrino.
    \end{itemize}
\end{rSection}


\begin{rSection}{Publication}
    \begin{itemize}
        \item Inelastic nuclear scattering from neutrinos and dark matter \hfill \href{https://arxiv.org/pdf/2206.08590.pdf}{arxiv}\\
        {\footnotesize \it Bhaskar Dutta, \textbf{Wei-Chih Huang}, Jayden L. Newstead, Vishvas Pandey}
        \item Axion-Like Particle Production at Beam Dump Experiments with Distinct Nuclear Excitation Lines \hfill \href{https://arxiv.org/pdf/2207.13659.pdf}{arxiv}\\
        {\footnotesize \it Loyd Waites, Adrian Thompson, Adriana Bungau, Janet M. Conrad, Bhaskar Dutta, \textbf{Wei-Chih Huang}, Doojin Kim, Michael Shaevitz, Joshua Spitz}
    \end{itemize}
\end{rSection}


\begin{rSection}{Extra-Curricular Activities}
    \begin{itemize}
        \item Project Manager at \href{https://aggiecodingclub.com/}{Aggie Coding Club} \hfill Feb - Nov 2022
        \item Data Science Ambassador representing Physics Department at Texas A\&M \hfill 2022 - 2023
    \end{itemize}
\end{rSection}



%------------------------
% Use this more detailed section if you have Relevant work experience
% keep your resume to 1 page, if you need to remove a project to display relevant experience
% that is okay
% ----------------------------
% \begin{rSection}{EXPERIENCE}

% \textbf{Role Name} \hfill Jan 2017 - Jan 2019\\
% Company Name \hfill \textit{San Francisco, CA}
%  \begin{itemize}
%     \itemsep -3pt {}
%      \item Achieved X\% growth for XYZ using A, B, and C skills.
%      \item Led XYZ which led to X\% of improvement in ABC
%     \item Developed XYZ that did A, B, and C using X, Y, and Z.
%  \end{itemize}

% \textbf{Role Name} \hfill Jan 2017 - Jan 2019\\
% Company Name \hfill \textit{San Francisco, CA}
%  \begin{itemize}
%     \itemsep -3pt {}
%      \item Achieved X\% growth for XYZ using A, B, and C skills.
%      \item Led XYZ which led to X\% of improvement in ABC
%     \item Developed XYZ that did A, B, and C using X, Y, and Z.
%  \end{itemize}

% \end{rSection}


\begin{rSection}{Honors and Awards}
    \begin{itemize}
        \item \textbf{Three Years Tsing Hua University Scholarship (2\% acceptance rate)} \hfill 2015 - 2018 \\
        Tuition wavier plus accommodation and textbooks subsidy
        \item \textbf{Undergraduate Research Scholarship} \hfill Fall 2018 \\
        The scholarship for the New Gravity Theory
        \item \textbf{Data Science Ambassador Scholarship}\hfill 2022 - 2023\\
        Data Science Ambassador Scholarship Program at Texas A\&M Institute of Data Science
    \end{itemize}
\end{rSection}


\begin{rSection}{Teaching Experience}
    \begin{itemize}
        \item \textbf{Teaching Assistant} {{\it Texas A\&M} Thermodynamics and Statistical Mechanics} \hfill Fall 2019
        \item \textbf{Teaching Assistant} {{\it Texas A\&M} Electricity and Magnetism for Engineering and Science (Lab)} \hfill Summer 2020
        \item \textbf{Teaching Assistant} {{\it Texas A\&M} Newtonian Mechanics for Engineering and Science} \hfill Spring 2020 - present
        \item \textbf{Teaching Assistant} {{\it Texas A\&M} Electricity and Magnetism for Engineering and Science} \hfill Fall 2020 - present
    \end{itemize}
\end{rSection}


\end{rSection}

% \begin{rSection}{SKILLS}
% \begin{tabular}{ @{} >{\bfseries}l @{\hspace{6ex}} l }
% Languages & Python, Bash, Mathematica \\
% Tools & Git, Docker, AWS, Django, Pandas, Numpy, SciPy\\
% \\
% \end{tabular}\\
% \end{rSection}


\end{document}
